\chapter {Classifying Spaces}

\section {Introduction}

Let ${\cal G}$ be a simplicial group $0$--reduced. The software  {\tt Kenzo} allows 
the construction of the universal bundle $\bar{\cal W}{\cal G}$, called the 
classifying space\index{classifying space} of ${\cal G}$. 
It is known that this is the base space of  a principal bundle
such that the total space is contractible. We know also that if ${\cal G}$ is
an Abelian simplicial group then $\bar{\cal W}{\cal G}$ is also an Abelian
simplicial group, otherwise if ${\cal G}$ is non--Abelian, then $\bar{\cal W}{\cal G}$
is only a simplicial set. In dimension $n$ the simplices of the
classifying space are elements of the  product:
$${\cal G}_{n-1}\times {\cal G}_{n-2}\times\cdots\times{\cal G}_0,$$
where ${\cal G}_k$ is the set of the $k$--simplices of the given simplicial group. \par
A simplex of $\bar{\cal W}{\cal G}$ is called {\bf gbar}\index{gbar} ({\em geometric} bar) 
and  may be seen as an $n$--uple
$$(g_{n-1}, g_{n-2}, \ldots, g_1, g_0),$$
where the $g_i$'s are elements of ${\cal G}_i$, possibly {\em degenerated}.
In {\tt Kenzo} a gbar is represented externally under the form:\index{gbar!representation}
\begin{center}
{\tt <<GBar < $[\eta g]_{n-1}$><$[\eta g]_{n-2}$ > ...  < $[\eta g]_1$ > < - $g_0$> >>}
\end{center}
where  a term such as $[\eta g]_i$ is a simplex of ${\cal G}_i$. Let us recall that
our group ${\cal G}$ is assumed $0$--reduced; as a consequence, the $g_0$ compound
of a gbar is necessarily the identity element of ${\cal G}$. \par
In the  symbolic expression $[\eta g]_i$,  the operator
$\eta$ represents a sequence (possibly the void one) of degeneracy operators
applied coherently upon some element of ${\cal G}_k$, for $k\leq i$. For instance $[\eta g]_1$
can be $\eta_0 g_0$, i.e. the $0$--degeneracy of the base point and $[\eta g]_2$ can be
can be $\eta_1\eta_0 g_0$. To be more explicit, let us take the Abelian simplicial group
$K({\Z}_2, 1)$ (see the simplicial groups chapter), then some elements of the basis in dimension $4$ of
its classifying space are, for instance:
{\footnotesize\begin{verbatim}
<<GBar<- 3><- 2><- 1><- 0>>>
<<GBar<- 3><1-0 0><0 0><- 0>>>
<<GBar<1 2><- 2><0 0><- 0>>>
<<GBar<1 2><0 1><0 0><- 0>>>
<<GBar<2 2><1 1><0 0><- 0>>>
<<GBar<2-0 1><1 1><0 0><- 0>>>
<<GBar<2-1 1><1 1><- 1><- 0>>>
<<GBar<2-1 1><0 1><0 0><- 0>>>
\end{verbatim}}

We recall that the symbol {\tt -}, in front of an element means ``no degeneracy''.

\section {Face, degeneracy and group operations}

Let us denote a gbar under the following generic form:
$$g_{bar}=(g_{n-1}, g_{n-2}, \ldots, {\bf g_j}, \ldots, g_1, g_0).$$
In this symbolism, $g_0$ is also the base point of the initial simplicial group.
The rule for the face operators is the following:
\begin{itemize}
\item If $j=n$, $\delta_n g_{bar}= (g_{n-2}, g_{n-3}, \ldots, g_1, g_0).$
\item If $j \not=n$, 
  $\delta_j g_{bar}= (\delta_j g_{n-1}, \ldots, {\bf \delta_j g_{j+1}, g_{j-1}.\delta_j g_j, g_{j-1}}, \ldots, g_1, g_0),$
where in the term $\bf g_{j-1}.\delta_j g_j$, the dot represents the group composition.
\item If $j=0$, $\delta_0 g_{bar}= (\delta_0 g_{n-1}, \ldots, \delta_0 g_1).$
\end{itemize}
The rule for the degeneracy operators is:
$$\eta_j g_{bar}= 
 (\eta_j g_{n-1}, \ldots, {\bf \eta_j g_{j+1}, \eta_{j-1}\eta_{j-2}\ldots\eta_1\eta_0 g_0, g_{j-1}}, \ldots, g_1, g_0).$$
If ${\cal G}$ is Abelian then $\bar{\cal W}{\cal G}$ is also  Abelian  and its
composition law is simply:
$$(g_{n-1}, \ldots, g_1, g_0).(g'_{n-1}, g'_1, g'_0)= 
  (g_{n-1}.g'_{n-1},  \ldots, g_1.g'_1, g_0.g'_0).$$
It is well known that if the initial group is not Abelian then, in general, the previous
composition law is not compatible with the simplicial structure.

\section {The functions for the classifying spaces}

{\parindent=0mm
{\leftskip=5mm
{\tt gbar} {\em dmns dop-1 elem-1 ... dop-n elem-n} \hfill {\em [Macro]} \par}
{\leftskip=15mm
Build a gbar in dimension {\em dmns}. The sequence {\em dop-1 elem-1 ... dop-n elem-n}, 
is a sequence of pairs ({\em dop-k, elem-k}), where {\em dop-k} is  a {\bf binary coded}
degeneracy operator and {\em elem-k} a simplex of the initial simplicial group. The ordering of
the pairs in the sequence must conform to the theoritical definition of a gbar. 
If the sequence of pairs is void, the special call {\tt (gbar 0)}, 
creates the base point of the classifying space
externally represented by {\tt <<GBAR>>}. This base point can be reached through 
the system symbol {\tt +null-gbar+}. \par}
{\leftskip=5mm
{\tt classifying-space-cmpr} {\em cmpr}\hfill {\em [Function]} \par}
{\leftskip=15mm
From a comparison function {\em cmpr} defined for a simplicial group ${\cal G}$, build a
comparison function for the classifying space of ${\cal G}$. \par}
{\leftskip=5mm
{\tt classifying-space-basis} {\em basis}\hfill {\em [Function]} \par}
{\leftskip=15mm
From the basis function {\em basis} of a simplicial group ${\cal G}$, build a
basis function for the classifying space of ${\cal G}$. If ${\cal G}$ is locally effective,
this function returns the symbol {\tt :locally-effective}.\par}
{\leftskip=5mm
{\tt classifying-space-face} {\em face sintr-grml}\hfill {\em [Function]} \par}
{\leftskip=15mm
From the face function {\em face} and the lisp function for the group law, {\em sintr-grml},
of a simplicial group ${\cal G}$, build a
face function for the classifying space of ${\cal G}$. \par}
{\leftskip=5mm
{\tt classifying-space} {\em smgr}\hfill {\em [Method]} \par}
{\leftskip=15mm
Build the simplicial set, classifying space of the simplicial group {\em smgr}.
The construction uses the three preceding auxiliary functions. The base point
of the returned simplicial set is the null gbar {\tt +null-gbar+}. \par}
{\leftskip=5mm
{\tt classifying-space} {\em A-smgr}\hfill {\em [Method]} \par}
{\leftskip=15mm
Build the Abelian simplicial group, classifying space of the Abelian simplicial group {\em A-smgr}.
The group law and its inverse (slots {\tt :sintr-grml} and {\tt :sintr-grin}) are built
by the two auxiliary functions: \par}
{\leftskip=20mm
{\tt classifying-space-grml-sintr} and \par}
{\leftskip=20mm
{\tt classifying-space-grin-sintr}, \par}
{\leftskip=15mm
defining the laws of this new group from the corresponding laws of {\em A-smgr}. \par}
}

\section {Eilenberg-Mac Lane spaces $K(\pi,n)$}

We\index{$K(\pi,n)$} have already described $K({\Z},1)$ and $K({\Z}_2,1)$ in the chapter on simplicial groups. 
This section gives some complements about the construction of $K({\Z},n)$ and $K({\Z}_2,n)$. The trivial case
for $n=0$ is not implemented in {\tt Kenzo}. For $n > 1$, these simplicial groups
are built recursively as the classifying spaces of $K({\Z},n-1)$ and $K({\Z}_2,n-1)$, respectively, as shown
in the following description:
\par
\vskip 0.40cm
{\parindent=0mm
{\leftskip=5mm
{\tt k-z} {\em n} \hfill {\em [Function]} \par}
{\leftskip=15mm
Build $K({\Z},n)$ recursively from $K({\Z},n-1)$. If $n=1$, return $K({\Z},1)$
{\footnotesize\begin{verbatim}
     (defun K-Z (n)
        (the ab-simplicial-group
           (if (= n 1)
               (k-z-1)
               (classifying-space (k-z (1- n))))))
\end{verbatim}}
\par}
{\leftskip=5mm
{\tt k-z2} {\em n} \hfill {\em [Function]} \par}
{\leftskip=15mm
Build $K({\Z}_2,n)$ recursively from $K({\Z}_2,n-1)$. If $n=1$, return $K({\Z}_2,1)$
{\footnotesize\begin{verbatim}
     (defun K-Z2 (n)
        (the ab-simplicial-group
           (if (= n 1)
               (k-z2-1)
               (classifying-space (k-z2 (1- n))))))
\end{verbatim}}
\par}}
These spaces have an effective homology which is computed via a {\em search-efhm} method (described in a next chapter).
For the case $n=1$, we may recall here that $K({\Z},1)$, though being locally effective, 
has the homology of $S^1$. The  {\em search-efhm} method applied to this chain complex does the following:  a reduction
between $K({\Z},1)$ and the chain complex {\tt circle} (described in the chain complex chapter) is
built using the function {\tt kz1-rdct} and the homotopy equivalence, value of the slot {\tt efhm}, is
built between this reduction and the trivial reduction of $K({\Z},1)$ upon itself.\par
On the other hand, $K({\Z},2)$ has a very simple finite basis in every dimension (see $\R$-projective 
spaces), so its homology is computed directly from its basis. The  homotopy 
equivalence, value of the slot {\tt efhm}, is the trivial homotopy equivalence of $K({\Z},2)$ upon
itself. 
\newpage

\subsection* {Examples}

Let us begin by showing some examples of gbars built  from various
initial simplicial groups. In the second statement, the initial simplicial group
is $K({\Z}_2, 1)$, in the third, it is $\Omega({\em Moore}(2,2))$. We recall that
in the call of the function {\tt gbar}, the degeneracy operators appear as integers,
according to the general rule in {\tt Kenzo}, but the printed result shows in clear
the sequence of $\eta_i$'s.
{\footnotesize\begin{verbatim}
(gbar 2 1 'a 2 'b)  ==>

<<GBar<0 A><1 B>>>

(gbar 4 0 3 3 0 1 0 0 0)  ==>

<<GBar<- 3><1-0 0><0 0><- 0>>>

(setf gbar-mr22 (gbar 4 0 (loop3 3 'm2 1 4 'n3 1)
                        0 (loop3 0 'n3 1)
                        0 (loop3 0 'm2 1)
                        0 +null-loop+))   ==>

<<GBar<- <<Loop[1-0 M2][2 N3]>>><- <<Loop[N3]>>><- <<Loop[M2]>>>
      <- <<Loop>>>>>
\end{verbatim}}
Let us test the face function upon this simplex. The user will note
that the face function returns an abstract simplex.
{\footnotesize\begin{verbatim}
(setf om (loop-space (moore 2 2)))  ==>

[K18 Simplicial-Group]

(setf face (classifying-space-face (face om) (sintr (grml om))))  ==>

#<Closure (FLET CLASSIFYING-SPACE-FACE RSLT) @ #x49af32>

(dotimes (i 5) (print (funcall face i 4 gbar-m22)))  ==>

<AbSm - <<GBar<- <<Loop[0 M2][1 M2]>>><- <<Loop[M2]>>><- <<Loop>>>>>> 
<AbSm 0 <<GBar<- <<Loop[M2]>>><- <<Loop>>>>>> 
<AbSm - <<GBar<- <<Loop[0 M2][N3]>>><- <<Loop[M2\2]>>><- <<Loop>>>>>> 
<AbSm - <<GBar<- <<Loop[N3\2]>>><- <<Loop[M2]>>><- <<Loop>>>>>> 
<AbSm - <<GBar<- <<Loop[N3]>>><- <<Loop[M2]>>><- <<Loop>>>>>> 
\end{verbatim}}
Let us work now, directly with the classifying space of $K({\Z}_2, 1)$.
{\footnotesize\begin{verbatim}
(setf cs-kz21 (classifying-space (k-z2-1)))  ==>

[K13 Abelian-Simplicial-Group]
\end{verbatim}}
\newpage
{\footnotesize\begin{verbatim}
(orgn cs-kz21)  ==>

(CLASSIFYING-SPACE [K1 Abelian-Simplicial-Group])

(setf elem-1 (first (basis cs-kz21 4)))  ==>

<<GBar<- 3><- 2><- 1><- 0>>>

(? cs-kz21 4 elem-1)  ==>

----------------------------------------------------------------------{CMBN 3}
<2 * <<GBar<- 2><- 1><- 0>>>>
------------------------------------------------------------------------------

(? cs-kz21 *)  ==>

----------------------------------------------------------------------{CMBN 2}
------------------------------------------------------------------------------

(cprd cs-kz21  4 elem-1)  ==>

----------------------------------------------------------------------{CMBN 4}
<1 * <TnPr <<GBar>> <<GBar<- 3><- 2><- 1><- 0>>>>>
<1 * <TnPr <<GBar<- 1><- 0>>> <<GBar<- 1><- 0>>>>>
<1 * <TnPr <<GBar<- 3><- 2><- 1><- 0>>> <<GBar>>>>
------------------------------------------------------------------------------

(dotimes (i 5)
    (print (face cs-kz21 i 4 elem-1)))  ==>

<AbSm - <<GBar<- 2><- 1><- 0>>>> 
<AbSm 0 <<GBar<- 1><- 0>>>> 
<AbSm 1 <<GBar<- 1><- 0>>>> 
<AbSm 2 <<GBar<- 1><- 0>>>> 
<AbSm - <<GBar<- 2><- 1><- 0>>>> 
\end{verbatim}}
Let us test the law group with $K({\Z}, 1)$. We recall the simplices
of  this Abelian simplicial group are represented as lists of integers.
The functions {\tt grml} and {\tt grin} accept also the gbars under the form
of abstract simplices (see the last statement).
{\footnotesize\begin{verbatim}
(setf cs-kz1 (classifying-space (k-z-1)))  ==>

[K13 Abelian-Simplicial-Group]

(grml cs-kz1 3 (crpr 0 (gbar 3 0 '(1 2) 0 '(3) 0 '())
                     0 (gbar 3 0 '(-1 -2) 0 '(-3) 0 '())))  ==>

<AbSm 2-1-0 <<GBar>>>

(grml cs-kz1 3 (crpr 0 (gbar 3 0 '(1 2) 0 '(3) 0 '())
                     4 (gbar 2 0 '(-3) 0 '())))             ==>

<AbSm - <<GBar<- (1 2)><0 NIL><- NIL>>>>

(grml cs-kz1 3 (crpr 0 (gbar 3 0 '(1 2) 0 '(3) 0 '())
                     1 (gbar 2 0 '(-3) 0 '())))             ==>

<AbSm - <<GBar<- (1 -1)><- (3)><- NIL>>>>

(grin cs-kz1 3 (gbar 3 0 '(1 2) 1 '() 0 '()))  ==>

<AbSm - <<GBar<- (-1 -2)><0 NIL><- NIL>>>>

(grin cs-kz1 3 *)  ==>

<AbSm - <<GBar<- (1 2)><0 NIL><- NIL>>>>
\end{verbatim}}
As $K({\Z}, 1)$ and $K({\Z_2}, 1)$ are Abelian simplicial groups, we may iterate the classifying
space construction and retrieve some known results.
{\footnotesize\begin{verbatim}
(cat-init)    ;; re-initialization

(setf k-z-3 (k-z 3))  ==>

[K25 Abelian-Simplicial-Group]

(homology k-z-3 0 10)  ==>

Homology in dimension 0 :

Component Z

Homology in dimension 1 :

Homology in dimension 2 :

Homology in dimension 3 :

Component Z

Homology in dimension 4 :

Homology in dimension 5 :

Component Z/2Z

Homology in dimension 6 :

Homology in dimension 7 :

Component Z/3Z

Homology in dimension 8 :

Component Z/2Z

Homology in dimension 9 :

Component Z/2Z

(setf k-z2-5 (k-z2 5))  ==>

[K342 Abelian-Simplicial-Group]

(homology k-z2-5 0 7)  ==>

Homology in dimension 0 :

Component Z

Homology in dimension 1 :

Homology in dimension 2 :

Homology in dimension 3 :

Homology in dimension 4 :

Homology in dimension 5 :

Component Z/2Z

Homology in dimension 6 :

--done--
\end{verbatim}}

\subsection* {Lisp files concerned in this chapter}

{\tt classifying-space.lisp}, {\tt k-pi-n.lisp}.
