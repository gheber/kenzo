\chapter {Serre spectral sequence}

\section {Introduction}

This chapter is devoted to the {\tt Kenzo} implementation of the
Serre spectral sequence\index{Serre spectral sequence} 
with the aim to compute the homology groups
of the total  space of a fibration. From the programming point of view,
this method will be used according to our general scheme of searching homology 
(generic function {\tt search-efhm}) applied  to an object of type {\tt fibration-total}.

\section {The topological problem}

Let ${\cal G}$ be a simplicial group (the fiber space),  $B$, a $1$--reduced simplicial set (the base space) and
$\tau$ (the fibration), a simplicial morphism of degree $-1$, $\tau: B \longrightarrow {\cal G}$.
We know, from a previous chapter (fibrations) that the software {\tt Kenzo} knows how to build
the total space of the fibration: $B \times_\tau {\cal G}$, which has the same simplices as
$B \times {\cal G}$ and the same face operators except the last one, given by:
$$\partial_n(b,g)  =  (\partial_n b, \tau(b).\partial_n g), \quad b\in B, g \in {\cal G},$$
where the '{\tt .}' denotes the group operation in ${\cal G}$. This discrepancy between the respective face operators
$\partial_n$ induces a discrepancy between the differential operators. We suppose now, that 
the base and  fiber  spaces are objects of effective homology type, i.e. there exists homotopy equivalences
$$\diagram{
  & \widehat{B}  & & & & \widehat{\cal G} & & \cr
 \rho_1 \swarrow\nearrow  & & \searrow\nwarrow \rho_2 &\quad & \rho'_1\swarrow\nearrow & & \searrow\nwarrow\rho'_2 & \cr
B  & & EB & & {\cal G}   & & EG \cr
           }$$

where the chain complexes $EB$ and $EG$ are effective. The problem consists in finding a homotopy equivalence
between $B \times_\tau {\cal G}$ and an effective chain complex noted $EB \widetilde{\otimes} EG$; 
the underlying graded module of $EB \widetilde{\otimes} EG$ is $EB \otimes EG$ but the differential
is twisted in a rather complicated way. In fact, the chain complex $EB \widetilde{\otimes} EG$ is nothing
but the Hirsh complex\footnote{{\bf Guy Hirsh.} {\em Sur les groupes d'homologie des espaces fibr\'es},
Bulletin de la Soci\'et\'e Math\'ematique de Belgique, 1954, vol. 6, pp. 79-96.}.

\subsection {The Ronald Brown reduction}

We start from the Eilenberg-Zilber reduction  (built in {\tt Kenzo} by the function {\tt ez}).
$$
\diagram{
{{\cal C}_*(B \times {\cal G})} & \stackrel{h}{\longrightarrow} & {}^s{{\cal C}_*(B \times {\cal G})} \cr
 {\scriptstyle f} \downarrow \uparrow {\scriptstyle g}  \cr
 {{\cal C}_*(B) \otimes {\cal C}_*({\cal G})} \cr
}.
$$
But, in fact we are interested in the following one (the Ronald Brown reduction):
$$
\diagram{
{{\cal C}_*(B \times_\tau {\cal G})} & \stackrel{h}{\longrightarrow} & {}^s{{\cal C}_*(B \times_\tau {\cal G})} \cr
 {\scriptstyle f} \downarrow \uparrow {\scriptstyle g}  \cr
 {{\cal C}_*(B) \otimes_t {\cal C}_*({\cal G})} \cr
}
$$
where the symbol $\otimes_t$ represents a {\rm twisted tensor product}, induced by the twisting
operator $\tau$; this twisting operator is nothing but the  Shih twisting cochain\footnote{{\bf Gugenheim.},
  {\em On the chain complex of a fibration}.
   Illinois Journal of Mathematics, 1972, vol. 16, pp. 398-414. } determined by Szczarba\footnote
{{\bf R. H. Szczarba.}
  {\em The homology of twisted cartesian products}.
   Transactions of the American Mathematical Society,
   1961, vol. 100, pp. 197-216.}. Here, to obtain the Szczarba cochain, we apply the perturbation lemma (Ronnie Brown).
In effect, writing 
$$ d_\tau = d + (d_\tau -d),$$
one may consider that the differential $d_\tau$ of $B \times_\tau {\cal G}$ 
is the differential $d$ of $B \times {\cal G}$ modified by the perturbation $d_\tau - d$.
The perturbation morphism $d_\tau -d$ is deduced from the  difference  between both  $\partial_n$ face operators. 
This being done,  a perturbation may possibly be 
propagated along the Eilenberg-Zilber reduction  (see the method {\tt add} applied to a reduction)
and  the perturbation lemma  gives also, as a bonus, the perturbation to be applied to the dif\-fe\-ren\-ti\-al 
of $B \otimes {\cal G}$  to obtain the differential of $B \otimes_t {\cal G}$. 

\subsection {The Gugenheim algorithm}

On the other hand, from the homotopy equivalences
$$\diagram{
  & \widehat{B} & & & & \widehat{\cal G} & & \cr
 \rho_1 \swarrow\nearrow  & & \searrow\nwarrow \rho_2 &\quad & \rho'_1\swarrow\nearrow & & \searrow\nwarrow\rho'_2 & \cr
B  & & EB & & {\cal G}   & & EG \cr
           }$$
it is possible to build their tensor product, i.e. the   homotopy equivalence:
$$\diagram{
  & \widehat{B}\otimes \widehat{\cal G} &    \cr
 \swarrow\nearrow & & \searrow\nwarrow \cr
B\otimes {\cal G}  & & EB \otimes EG \cr}
$$
This is done in {\tt Kenzo} by the method  {\tt tnsr-prdc}.
In fact, we are interested in the following one:
$$\diagram{
  & \widehat{B}\widetilde{\otimes}\widehat{\cal G} &    \cr
 \swarrow\nearrow & & \searrow\nwarrow \cr
B\otimes_t {\cal G}  & & EB \widetilde{\otimes} EG \cr}
$$
where $\widehat{B}\widetilde{\otimes}\widehat{\cal G}$ and $EB \widetilde{\otimes} EG$ are appropriate
strongly twisted tensor products determined by the perturbation lemma: this last homotopy equivalence
is obtained by propagating over the previous one, the perturbation to be applied to the differential 
of $B \otimes {\cal G}$  to obtain the differential of $B \otimes_t {\cal G}$.
The homotopy equivalence so obtained is called in {\tt Kenzo} the {\em right Serre homotopy equivalence}.

\subsection {Assembling the puzzle}

Let us build now the so-called {\em left Serre homotopy equivalence}, 
$$\diagram{
  & B\times_\tau{\cal G} &    \cr
 \swarrow\nearrow & & \searrow\nwarrow \cr
B\times_\tau {\cal G}  & & B\otimes_t {\cal G} \cr}
$$
where the left reduction is the trivial one and the right reduction is
the Brown reduction. We may then compose both Serre homotopy equivalences to obtain
what we were looking for: 
$$\diagram{
  & B\times_\tau {\cal G}  & & &\widehat{B}\otimes_t\widehat{\cal G} & & \cr
 \rho_1 \swarrow\nearrow  & & \searrow\nwarrow \rho_2 & \rho'_1\swarrow\nearrow & & \searrow\nwarrow\rho'_2 & \cr
B\times_\tau {\cal G}  & &B\otimes_t {\cal G} &B\otimes_t {\cal G} & &EB \otimes_t EG \cr
           }$$

\section {The functions for the Serre spectral sequence}

The following functions are given for information purpose.
\vskip 0.35cm
{\parindent=0mm
{\leftskip=5mm
{\tt fibration-dtau-d-intr} {\em fibration}  \hfill {\em [Function]} \par}
{\leftskip=15mm
Build the lisp function corresponding to the difference $d_\tau - d$, where $d$ is the differential
in $B \times {\cal G}$ and $d_\tau$ the differential in $B \times_\tau {\cal G}$. This is an internal
function and is used in the following one. \par}
{\leftskip=5mm
{\tt fibration-dtau-d} {\em fibration}  \hfill {\em [Function]} \par}
{\leftskip=15mm
Build the morphism $d_\tau-d$, which is the perturbation morphism to be added to
the differential  $d$ of $B \times {\cal G}$  to obtain  $d_\tau$, 
the differential in $B \times_\tau {\cal G}$. We recall that both simplicial sets have
the same simplices. The source and the target of this morphism is the total space
$B \times_\tau {\cal G}$ and the degree is $-1$. \par}
{\leftskip=5mm
{\tt Brown-reduction} {\em fibration}  \hfill {\em [Function]} \par}
{\leftskip=15mm
Return two values: a) the Brown reduction and b) the  perturbation to be applied to the differential of the tensor
product $B \otimes {\cal G}$  to obtain the differential of the twisted tensor
product $B \otimes_t {\cal G}$. \par}
{\leftskip=5mm
{\tt left-Serre-efhm} {\em fibration}  \hfill {\em [Function]} \par}
{\leftskip=15mm
Build the left Serre homotopy equivalence from the argument {\em fibration}. \par}
{\leftskip=5mm
{\tt right-Serre-efhm} {\em fibration}  \hfill {\em [Function]} \par}
{\leftskip=15mm
Build the right Serre homotopy equivalence from the argument {\em fibration}. \par}
}

\section {Searching homology for fibration total spaces}

The comment list of a {\tt Kenzo} object which is the total space of a fibration has the form
\index{searching homology!fibration total space}
{\tt (fibration-total {\em fibration})}. The {\tt search-efhm} method applied to a simplicial
set having this kind of comment list, builds the composition of the left and right Serre homotopy
equivalences. The right one needs the homotopy equivalences attached respectively to the
base and  fiber spaces. This may involve a recursive call of {\tt search-efhm}.
If the base space or the fiber space
are locally effective, the total space is locally effective and the method
cannot compute the homology.
{\footnotesize\begin{verbatim}
(defmethod SEARCH-EFHM (smst (orgn (eql 'fibration-total)))
  (declare
     (type simplicial-set smst))
  (the homotopy-equivalence
     (fibration-total-efhm (second (orgn smst)))))

(defun FIBRATION-TOTAL-EFHM (fibration)
  (declare (type fibration fibration))
  (the homotopy-equivalence
       (cmps (left-serre-efhm fibration)
             (right-serre-efhm fibration))))
\end{verbatim}}

\subsection* {Examples}

We begin by some examples similar to examples that we have seen in the fibrations chapter.
For the user, the delicate point is to write in Lisp a correct twisting operator. There is no check
in this version of {\tt Kenzo} for the coherency of the twisting operator. In the first example
we define a fibration $\tau: S^2 \rightarrow K({\Z},1)$ with $\tau(s2)= (2)$, then we build the
total space of the fibration, namely $P^3{\R}$, and we compute some known homology groups.
{\footnotesize\begin{verbatim}
(setf sph2 (sphere 2))  ==>

[K1 Simplicial-Set]

(setf kz1 (k-z-1))  ==>

[K6 Abelian-Simplicial-Group]

(setf tw2 (build-smmr
             :sorc sph2
             :trgt kz1
             :degr -1
             :sintr #'(lambda (dmns gmsm)
                        (unless (= dmns 2)
                                (error "Dimension error for 
                                        the twisting operator S2-->KZ1"))
                        (absm 0 (list 2)))
	     :orgn '(s2-tw-kz1)))            ==>

[K18 Fibration]

(? tw2 2 's2)  ==>

<AbSm - (1)>

(? tw2 0 '*)  ==>

Error: Dimension error for the twisting operator S2-->KZ1

(setf p3r (fibration-total tw2))  ==>

[K24 Simplicial-Set]

(homology p3r 0 10)   ==>

Homology in dimension 0 :

Component Z

Homology in dimension 1 :

Component Z/2Z

Homology in dimension 2 :

Homology in dimension 3 :

Component Z

Homology in dimension 4 :

Homology in dimension 5 :

Homology in dimension 5 :

Homology in dimension 7 :

Homology in dimension 8 :

Homology in dimension 9 :

--done--
\end{verbatim}}
Let us do the same kind of computations with $\tau(s2)= (5)$ and $\tau(s2)= (17)$.
{\footnotesize\begin{verbatim}
(setf tw5 (build-smmr
             :sorc sph2
             :trgt kz1
             :degr -1
             :sintr #'(lambda (dmns gmsm)
                        (absm 0 (list 5)))
             :orgn '(s2-tw-kz1-5)))         ==>

[K110 Fibration]

(setf total-5 (fibration-total tw5))  ==>

[K111 Simplicial-Set]

(homology total-5 0 5)

Homology in dimension 0 :

Component Z

Homology in dimension 1 :

Component Z/5Z

Homology in dimension 2 :

Homology in dimension 3 :

Component Z

Homology in dimension 4 :

(setf tw17 (build-smmr
             :sorc sph2
             :trgt kz1
             :degr -1
             :sintr #'(lambda (dmns gmsm)
                        (absm 0 (list 17)))
             :orgn '(s2-tw-kz1-17)))         ==>

[K165 Fibration]

(setf total-17 (fibration-total tw17))  ==>

[K166 Simplicial-Set]

(homology total-17 0 4)  ==>

Homology in dimension 0 :

Component Z

Homology in dimension 1 :

Component Z/17Z

Homology in dimension 2 :

Homology in dimension 3 :

Component Z
\end{verbatim}}
\newpage
Using the last total space ({\tt total-17}), it is instructive to enter in the details
of the resulting homotopy equivalence that the system builds to compute the homology
groups. First, the function {\tt echcm} selects the {\bf effective} chain complex of the homotopy
equivalence, value of the slot {\tt efhm} of the simplicial set {\tt total-17}, and we may 
print some basis of this chain complex. In fact, all basis in dimension above $3$ are null.
The presence of the symbol {\tt S1} is due to the fact that the effective chain complex
used for the homology of $K({\Z},1)$ is the circle (see the classifying spaces chapter).
{\footnotesize\begin{verbatim}
(setf echcm (echcm total-17))  ==>

[K205 Chain-Complex]

(dotimes (i 6) (print (basis echcm i)))  ==>

(<TnPr * *>) 
(<TnPr * S1>) 
(<TnPr S2 *>) 
(<TnPr S2 S1>) 
NIL
NIL
\end{verbatim}}
The resulting homotopy equivalence built by the system, may be summarized
by the diagram:

$$\diagram{
  &\widehat{S^2\times_\tau K({\Z},1)} &    \cr
 f\swarrow\nearrow & & \searrow\nwarrow g \cr
S^2\times_\tau K({\Z},1) & & [K205\, Chain\, Complex] \cr}
$$

where {\em f} and {\em g} are respectively the {\em descending} and {\em ascending}
morphisms of the left and right reductions of the homotopy equivalence. We recall that
these morphisms are obtained respectively by the functions {\tt lf} and {\tt rg}. We may
compose these morphisms to create a morphism {\tt fg} from $[K205]$ to $S^2\times_\tau K({\Z},1)$ and
see how it acts on the basis  elements in dimension $1$ and $2$:
{\footnotesize\begin{verbatim}
(setf fg (cmps (lf (efhm total-17)) (rg (efhm total-17))))  ==>

[K220 Morphism (degree 0)]

(setf gen1 (first (basis echcm 1)))  ==>

<TnPr * S1>

(? fg 1 gen1)  ==>

----------------------------------------------------------------------{CMBN 1}
<-1 * <CrPr 0 * - (1)>>
------------------------------------------------------------------------------

(setf gen2 (first (basis echcm 2)))  ==>

<TnPr S2 *>

(setf fg-gen2 (? fg 2 gen2))  ==>

----------------------------------------------------------------------{CMBN 2}
<-1 * <CrPr - S2 1-0 NIL>>
<-1 * <CrPr 1-0 * - (1 1)>>
<-1 * <CrPr 1-0 * - (1 2)>>
<-1 * <CrPr 1-0 * - (1 3)>>
<-1 * <CrPr 1-0 * - (1 4)>>
<-1 * <CrPr 1-0 * - (1 5)>>
<-1 * <CrPr 1-0 * - (1 6)>>
<-1 * <CrPr 1-0 * - (1 7)>>
<-1 * <CrPr 1-0 * - (1 8)>>
<-1 * <CrPr 1-0 * - (1 9)>>
<-1 * <CrPr 1-0 * - (1 10)>>
<-1 * <CrPr 1-0 * - (1 11)>>
<-1 * <CrPr 1-0 * - (1 12)>>
<-1 * <CrPr 1-0 * - (1 13)>>
<-1 * <CrPr 1-0 * - (1 14)>>
<-1 * <CrPr 1-0 * - (1 15)>>
<-1 * <CrPr 1-0 * - (1 16)>>
------------------------------------------------------------------------------
\end{verbatim}}
The boundary of this combination in $S^2\times_\tau K({\Z},1)$ is 
{\footnotesize\begin{verbatim}
(? total-17 *)

----------------------------------------------------------------------{CMBN 1}
<-17 * <CrPr 0 * - (1)>>
------------------------------------------------------------------------------
\end{verbatim}}
to which corresponds in the effective chain complex, the boundary of the basis
element in dimension $2$, $gen2=s2 \otimes *$:
{\footnotesize\begin{verbatim}
(? echcm 2 gen2)  ==>

----------------------------------------------------------------------{CMBN 1}
<17 * <TnPr * S1>>
------------------------------------------------------------------------------
\end{verbatim}}
The printing of the $3$ faces of the $17$ simplices of the combination
{\tt fg-gen2} shows the geometrical organisation of the twisted product:
{\footnotesize\begin{verbatim}
(dolist(iterm (cmbn-list fg-gen2))
       (dotimes (i 3)(print (face total-17 i 2 (gnrt iterm)))) 
                     (terpri))  ==>

<AbSm 0 <CrPr - * - NIL>> 
<AbSm 0 <CrPr - * - NIL>> 
<AbSm - <CrPr 0 * - (17)>> 

<AbSm - <CrPr 0 * - (1)>> 
<AbSm - <CrPr 0 * - (2)>> 
<AbSm - <CrPr 0 * - (1)>> 

<AbSm - <CrPr 0 * - (2)>> 
<AbSm - <CrPr 0 * - (3)>> 
<AbSm - <CrPr 0 * - (1)>> 

<AbSm - <CrPr 0 * - (3)>> 
<AbSm - <CrPr 0 * - (4)>> 
<AbSm - <CrPr 0 * - (1)>> 

<AbSm - <CrPr 0 * - (4)>> 
<AbSm - <CrPr 0 * - (5)>> 
<AbSm - <CrPr 0 * - (1)>> 

<AbSm - <CrPr 0 * - (5)>> 
<AbSm - <CrPr 0 * - (6)>> 
<AbSm - <CrPr 0 * - (1)>> 

<AbSm - <CrPr 0 * - (6)>> 
<AbSm - <CrPr 0 * - (7)>> 
<AbSm - <CrPr 0 * - (1)>> 

<AbSm - <CrPr 0 * - (7)>> 
<AbSm - <CrPr 0 * - (8)>> 
<AbSm - <CrPr 0 * - (1)>> 

<AbSm - <CrPr 0 * - (8)>> 
<AbSm - <CrPr 0 * - (9)>> 
<AbSm - <CrPr 0 * - (1)>> 
\end{verbatim}}
\newpage
{\footnotesize\begin{verbatim}
<AbSm - <CrPr 0 * - (9)>> 
<AbSm - <CrPr 0 * - (10)>> 
<AbSm - <CrPr 0 * - (1)>> 

<AbSm - <CrPr 0 * - (10)>> 
<AbSm - <CrPr 0 * - (11)>> 
<AbSm - <CrPr 0 * - (1)>> 

<AbSm - <CrPr 0 * - (11)>> 
<AbSm - <CrPr 0 * - (12)>> 
<AbSm - <CrPr 0 * - (1)>> 

<AbSm - <CrPr 0 * - (12)>> 
<AbSm - <CrPr 0 * - (13)>> 
<AbSm - <CrPr 0 * - (1)>> 

<AbSm - <CrPr 0 * - (13)>> 
<AbSm - <CrPr 0 * - (14)>> 
<AbSm - <CrPr 0 * - (1)>> 

<AbSm - <CrPr 0 * - (14)>> 
<AbSm - <CrPr 0 * - (15)>> 
<AbSm - <CrPr 0 * - (1)>> 

<AbSm - <CrPr 0 * - (15)>> 
<AbSm - <CrPr 0 * - (16)>> 
<AbSm - <CrPr 0 * - (1)>> 

<AbSm - <CrPr 0 * - (16)>> 
<AbSm - <CrPr 0 * - (17)>> 
<AbSm - <CrPr 0 * - (1)>> 
\end{verbatim}}
\begin{center}
---------
\end{center}
We have seen that for the special case of loop spaces, there is a
canonical fibration and that the function {\tt twisted-crts-prdc}
builds the total space $X \times_\tau \Omega X$.
{\footnotesize\begin{verbatim}
(setf sph3 (sphere 3))  ==>

[K221 Simplicial-Set]

(setf tw3 (twisted-crts-prdc sph3))  ==>

[K243 Simplicial-Set]

(face tw3 3 3 (crpr 0 's3 7 +null-loop+))  ==>

<AbSm - <CrPr 1-0 * - <<Loop[S3]>>>>
\end{verbatim}}
It is of course possible to study, any other fibration, provided one
defines a correct fibration $\tau$. In the following example
the fibration $\tau: S^3 \longrightarrow \Omega S^3$, is defined
by 
$\tau(s3)= s3^{-2},$ where $s3^{-2}$ is a word in the Kan simplicial version
of the first loop space of $S^3$. In Kenzo, this object is created by the
statement:
{\footnotesize\begin{verbatim}
(absm 0 (loop3 0 's3 -2))  ==>

<AbSm - <<Loop[S3\-2]>>>

(setf tws3 (build-smmr   :sorc sph3
                         :trgt (loop-space sph3)
                         :degr -1
                         :sintr #'(lambda (dmns gmsm) (absm 0 (loop3 0 's3 -2)))
                         :orgn '(s3-tw-lps3)))     ==>

[K248 Fibration]

(setf total (fibration-total tws3))  ==>

[K249 Simplicial-Set]

(homology total 0 7)  ==>

Homology in dimension 0 :

Component Z

Homology in dimension 1 :

Homology in dimension 2 :

Component Z/2Z

Homology in dimension 3 :

Homology in dimension 4 :

Component Z/2Z

Homology in dimension 5 :

Homology in dimension 6 :

Component Z/2Z
\end{verbatim}}
\newpage

\subsection* {Lisp files concerned in this chapter}

{\tt serre.lisp}, {\tt searching-homology.lisp}.
