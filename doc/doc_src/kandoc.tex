\chapter {Programming the Kan theory}

\section {Introduction}

Let us recall some definitions in relation with the Kan simplicial sets.
\vskip 0.40cm
{\bf Definition 1.} A Kan ``hat''\index{Kan hat} of dimension $n$ and index $i$ is a collection
of $n$ simplices of dimension  $(n-1)$, noted 
$(\sigma_0, \ldots, \sigma_{i-1}, \sigma_{i+1}, \ldots, \sigma_n)$,
such that the following conditions are fulfilled:
$$ \partial_j\sigma_k=\partial_k\sigma_{j+1},\quad k \not= i,\quad j+1 \not= i, \quad 0 \leq j \leq n-1, \quad
  0 \leq k \leq n-1, \quad k \leq j. $$
Note that there is no $\sigma_i$ term in the list above; in a sense, the Kan process consists in
constructing the missing $\sigma_i$ to be considered as a ``composition'' of the given $\sigma_j$'s.
\subsection* {Example}

We are going to work with the simplices of $\Delta^{\N}$. The set $(\sigma_1, \sigma_2) = $ {\tt ((0 2), (0 1))}
is a Kan hat of dimension $2$ and index $0$. In effect, $\delta_1\sigma_1=\delta_1\sigma_2 = {\tt (0)}$. 
Likewise, the two sets {\tt ((1 2), (0 1))} and {\tt ((1 2), (0 2))} are  Kan hats  of dimension $2$ and
of respective index $1$ and $2$. Of course, this definition applies equally if the simplices
are degenerate.
\vskip 0.40cm
{\bf Definition 2.} A {\em filling} of a Kan hat\index{Kan hat!filling},
$(\sigma_0, \ldots, \sigma_{i-1}, \sigma_{i+1}, \ldots, \sigma_n)$,
is an $n$--simplex $\tilde{\sigma}$, 
such that for every $j \not= i$, $\partial_j\tilde{\sigma}=\sigma_j.$

\subsection* {Example}

A filling of {\tt ((0 2), (0 1))} is the $2$-simplex {\tt (0 1 2)}. A filling of the Kan hat
{\tt ((1 1), (0 1))} is the degenerate $2$-simplex {\tt (0 1 1)}. A filling
of the vertex {\tt (0)} is the degenerate $1$--simplex {\tt (0 0)}.
\vskip 0.40cm
{\bf Definition 3.} A simplicial set is a Kan simplicial set if for every hat there exists a filling.

\subsection {Representation of a Kan simplicial set}

A Kan simplicial set is implemented as an instance\index{class!{\tt KAN}} of the class {\tt KAN}, 
subclass of the class {\tt SIMPLICIAL-SET}.
{\footnotesize\begin{verbatim}
(DEFCLASS KAN (simplicial-set)
  ((kfll :type kfll :initarg :kfll :reader kfll1)))
\end{verbatim}}
We recall that this new class inherits also from the class {\tt CHAIN-COMPLEX}. It has one
slot of its own:
\begin{description}
\item {{\tt kfll}}, an object of type {\tt KFLL}, in fact a lisp function with $3$ parameters describing
a Kan hat:
an index (an integer), a dimension (an integer) and the Kan hat (a list of abstract simplices). This function
must return a filling of the Kan hat argument, i.e. an abstract simplex satisfying the theoritical
definition.
\end{description}
A printing method has been associated to the class {\tt KAN} and the external re\-pre\-sen\-ta\-ti\-on
of  an instance is a string like {\tt [K{\em n} Kan-Simplicial-Set]}, where {\em n} is the number plate of
the Kenzo object.

\subsection {Helpful functions on Kan simplicial sets}

{\parindent=0mm
{\leftskip=5mm
{\tt cat-init} \hfill {\em [Function]} \par}
{\leftskip=15mm
Clear in particular {\tt *Kan-list*}, the list of user created Kan simplicial sets  and reset
the global counter to $1$. \par}
{\leftskip=5mm 
{\tt Kan} {\em n} \hfill {\em [Function]}\par}
{\leftskip=15mm 
Retrieve in the list {\tt *Kan-list*} the Kan object instance whose the Kenzo identification  is $n$.
If it does not exist, return {\tt NIL}. \par}
{\leftskip=5mm 
{\tt kfll \&rest} {\em args} \hfill {\em [Macro]}\par}
{\leftskip=15mm 
With only one argument (a Kan instance) return the slot {\tt kfll} of this instance. With $4$ arguments
like {\tt (kfll {\em kan indx dmns hat})}, return a filling of the Kan hat {\em hat}, of dimension {\em dmns}
and of index {\em indx} by applying the filling function value of the slot {\tt kfll} of the Kan instance
{\em kan}. \par}
}
\newpage
{\parindent=0mm
{\leftskip=5mm 
{\tt smst-kan} {\em smst kfll} \hfill {\em [Function]}\par}
{\leftskip=15mm 
With the filling lisp function {\em kfll}, 
{\bf transform} the simplicial set {\em smst} in an object of type {\tt KAN} 
(in other word, {\em smst} is {\tt modified}).
This is the easiest way to build a Kan simplicial set. \par}
{\leftskip=5mm 
{\tt check-hat} {\em kan indx dmns hat} \hfill {\em [Function]}\par}
{\leftskip=15mm 
Useful verification function to check if the collection of simplices {\em hat} is a valid Kan hat
of dimension {\em dmns} and of index {\em indx} in the Kan simplicial set {\em kan}. In fact this
works equally if {\em kan} is a general simplicial set. \par}
{\leftskip=5mm 
{\tt check-kan} {\em kan indx dmns hat} \hfill {\em [Function]}\par}
{\leftskip=15mm 
Useful verification function to check if the collection of simplices {\em hat} is a valid Kan hat
of dimension {\em dmns} and of index {\em indx} in the Kan simplicial set {\em kan}. This verification
function applies the filling function of the instance {\em kan} to the argument {\em hat} and perform
the  verification of the faces relations upon the resulting $dmns$--simplex. \par}
}

\subsection* {Examples}

Let us take again the small examples of the introduction. First we  define a function {\tt dkfll}, 
a filling function suitable for a  Kan hat in $\Delta^{\N}$. The user will note that
in the abstract simplices the degeneracy operators and the geometric simplices are coded in binary.
{\footnotesize\begin{verbatim}
(defun dkfll (indx dmns hat)
     (cond ((= 1 dmns)
            (absm 1 (gmsm (first hat))))
           ((= 0 indx)
            (let ((del-1 (absm-int-ext (first hat)))
                  (del-2 (absm-int-ext (second hat))))
              (absm-ext-int
                 (cons (first del-1)
                       (cons (second del-2) (rest del-1))))))
           ((= 1 indx)
            (let ((del-0 (absm-int-ext (first hat)))
                  (del-2 (absm-int-ext (second hat))))
              (absm-ext-int
                 (cons (first del-2) del-0))))
           (t
            (let ((del-0 (absm-int-ext (first hat)))
                  (del-1 (absm-int-ext (second hat))))
              (absm-ext-int
                 (cons (first del-1) del-0))))))

DKFLL

(setf d (delta-infinity))  ==>

[K5 Simplicial-Set]

(smst-kan d #'dkfll)  ==>

[K5 Kan-Simplicial-Set]

(kfll d 0 1 (list (absm 0 1)))  ==>

<AbSm 0 1>

(kfll d 0 2 (list (absm 0 5) (absm 0 3)))  ==>

<AbSm - 7>

(kfll d 1 2 (list (absm 0 6) (absm 0 3)))  ==>

<AbSm - 7>

(kfll d 2 2 (list (absm 0 6) (absm 0 5)))  ==>

<AbSm - 7>

(kfll d 0 2 (list (absm 0 3) (absm 0 3)))  ==>

<AbSm 1 3>

(kfll d 1 2 (list (absm 1 2) (absm 0 3)))  ==>

<AbSm 1 3>

(check-hat d 1 2 (list (absm 1 2) (absm 0 3)))  ==>

T

(check-hat d 0 1 (list (absm 0 1)))  ==>

T

(check-kan d 0 1 (list (absm 0 1)))  ==>

---done---
\end{verbatim}}

More elaborate examples with Kan simplicial sets will be given later in the
loop spaces chapter.

\subsection* {Lisp files concerned in this chapter}

{\tt kan.lisp}.
\par
[{\tt classes.lisp }, {\tt macros.lisp}, {\tt various.lisp}].
